% resume.tex
% vim:set ft=tex spell:

\documentclass[10pt,letterpaper]{article}
\usepackage[letterpaper,margin=0.75in]{geometry}
\usepackage[utf8]{inputenc}
\usepackage{mdwlist}
\usepackage[T1]{fontenc}
\usepackage{textcomp}
\usepackage{tgpagella}
\pagestyle{empty}
\setlength{\tabcolsep}{0em}

% indentsection style, used for sections that aren't already in lists
% that need indentation to the level of all text in the document
\newenvironment{indentsection}[1]%
{\begin{list}{}%
	{\setlength{\leftmargin}{#1}}%
	\item[]%
}
{\end{list}}

% opposite of above; bump a section back toward the left margin
\newenvironment{unindentsection}[1]%
{\begin{list}{}%
	{\setlength{\leftmargin}{-0.5#1}}%
	\item[]%
}
{\end{list}}

% format two pieces of text, one left aligned and one right aligned
\newcommand{\headerrow}[2]
{\begin{tabular*}{\linewidth}{l@{\extracolsep{\fill}}r}
	#1 &
	#2 \\
\end{tabular*}}

% make "C++" look pretty when used in text by touching up the plus signs
\newcommand{\CPP}
{C\nolinebreak[4]\hspace{-.05em}\raisebox{.22ex}{\footnotesize\bf ++}}

% and the actual content starts here
\begin{document}

\begin{center}
{\LARGE \textbf{Elliot Z. Lin}}

Address\ \ \textbullet
\ \ Apt. No.\ \ \textbullet
\ \ City, State, ZIP
\\
Phone \ \textbullet
\ \ Email
\end{center}

\hrule
\vspace{-0.4em}
\subsection*{Education}

\begin{itemize}
	\parskip=0.1em

	\item 
	\headerrow
		{\textbf{University of California, Berkeley}}
		{\textbf{Berkeley, CA}}
	\\
	\headerrow
		{\emph{College of Engineering, Electrical Engineering and Computer Science}}
		{\emph{2015 -- 2019}}
	\begin{itemize*}
		\item Cumulative GPA: 3.807
		\item Relevant Coursework
		\begin{itemize*}
			\item Computer Science Courses: Data Structures and Algorithms, The Structure and Interpretation of Computer Programs, Discrete Mathematics and Probability Theory
			\item Electrical Engineering Courses: Designing Information Devices and Systems
		\end{itemize*}
	\end{itemize*}

\end{itemize}

\hrule
\vspace{-0.4em}
\subsection*{Projects/Experience}

\begin{itemize}
	\parskip=0.1em
	
	\item
	\headerrow
		{\textbf{Cal Solar Car Team}}
		{\emph{Fall 2015-present}}
	\begin{itemize*}
		\item Improve and maintain car telemetry graphical viewer interface using Java
		\item Implemented additional graphics to monitor car pedals, a console to display CAN messages from the car, and debugged backend CAN interpreter
		\item Solder surface-mount components to and debug in-house-designed boards
	\end{itemize*}
	
	\item
	\headerrow
		{\textbf{Java Swing Text Editor}}
		{\emph{Spring 2016}}
	\begin{itemize*}
		\item Designed and developed text editor from scratch using Java Swing libraries exclusively
		\item Utilized specific and custom data structures to implement common features such as word wrap, undo/redo, insertion, and cursor
	\end{itemize*}
	
	\item
	\headerrow
		{\textbf{Routing Web Application}}
		{\emph{Spring 2016}}
	\begin{itemize*}
		\item Implemented BearMaps, a Google Maps-like web application for the locale of UC Berkeley
		\item Used data from OpenStreetMap to build database of nodes and paths stored as a graph
		\item Implemented A* search algorithm to find the shortest path between two nodes
		\item Utilized Apache Maven to build and run web application on server hosted locally
	\end{itemize*}
	
	\item
	\headerrow
		{\textbf{Scheme Interpreter}}
		{\emph{Fall 2015}}
	\begin{itemize*}
		\item Developed a Scheme interpreter written in Python
	\end{itemize*}
	
	\item
	\headerrow
		{\textbf{Multiplayer Tetris}}
		{\emph{Spring 2015}}
	\begin{itemize*}
		\item Developed a multiplayer Tetris game with a group of two other classmates
		\item Utilize Java Swing graphics to paint game environment and Java to implement game back-end
		\item Implement multiplayer functionality through a series of network listeners to communicate messages across a local area network
	\end{itemize*}
	
	\item
	\headerrow
		{\textbf{Intern -- iServiceSoft, Milpitas, CA}}
		{\emph{June - August 2014}}
	\begin{itemize*}
		\item Programmed label printer SDK to read and print XML template files
		\item Developed aforementioned software for both Mac and Windows platforms.
		\item Programmed in Objective C, C++, and Java
		\item Wrote JNI Interface to link Java to C++ library
	\end{itemize*}

\end{itemize}

\hrule
\vspace{-0.4em}
\subsection*{Skills}

\begin{indentsection}{\parindent}
\hyphenpenalty=1000
\begin{description*}
	\item[Programming Languages:]
	Java, Python, Scheme, SQLite, \LaTeX, Objective-C, Swift, git
	\item[Open Source Contributions:]
	VLC Media Player
	\item[Language skills:] English (Native proficiency); Mandarin Chinese (Professional proficiency); French (Limited working proficiency)
\end{description*}
\end{indentsection}

\end{document}
