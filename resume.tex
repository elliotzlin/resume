% resume.tex
% vim:set ft=tex spell:

\documentclass[10pt,letterpaper]{article}
\usepackage[letterpaper,margin=0.75in]{geometry}
\usepackage[utf8]{inputenc}
\usepackage{mdwlist}
\usepackage[T1]{fontenc}
\usepackage{textcomp}
\usepackage{tgpagella}
\pagestyle{empty}
\setlength{\tabcolsep}{0em}

% indentsection style, used for sections that aren't already in lists
% that need indentation to the level of all text in the document
\newenvironment{indentsection}[1]%
{\begin{list}{}%
	{\setlength{\leftmargin}{#1}}%
	\item[]%
}
{\end{list}}

% opposite of above; bump a section back toward the left margin
\newenvironment{unindentsection}[1]%
{\begin{list}{}%
	{\setlength{\leftmargin}{-0.5#1}}%
	\item[]%
}
{\end{list}}

% format two pieces of text, one left aligned and one right aligned
\newcommand{\headerrow}[2]
{\begin{tabular*}{\linewidth}{l@{\extracolsep{\fill}}r}
	#1 &
	#2 \\
\end{tabular*}}

% make "C++" look pretty when used in text by touching up the plus signs
\newcommand{\CPP}
{C\nolinebreak[4]\hspace{-.05em}\raisebox{.22ex}{\footnotesize\bf ++}}

% and the actual content starts here
\begin{document}

\begin{center}
{\LARGE \textbf{Elliot Z. Lin}}

Address\ \ \textbullet
\ \ Apt. No.\ \ \textbullet
\ \ City, State, ZIP
\\
Phone \ \textbullet
\ \ Email
\end{center}

\hrule
\vspace{-0.4em}
\subsection*{Education}

\begin{itemize}
	\parskip=0.1em

	\item 
	\headerrow
		{\textbf{University of California, Berkeley}}
		{\textbf{Berkeley, CA}}
	\\
	\headerrow
		{\emph{Bachelor of Science, Electrical Engineering and Computer Science}}
		{\emph{Expected May 2019}}
	\begin{itemize*}
		\item Cumulative GPA: 3.729
		\item Current Coursework: Machine Learning, Internet Architecture, Robotics
		\item Completed Coursework: Data Structures, Algorithms, Artificial Intelligence, Machine Structures, Discrete Mathematics and Probability Theory, Designing Information Devices and Systems, Signals and Systems
		\item Honors: Tau Beta Pi National Engineering Honor Society, Eta Kappa Nu EECS Honor Society
	\end{itemize*}

\end{itemize}


\hrule
\vspace{-0.4em}
\subsection*{Skills}

\begin{indentsection}{\parindent}
\hyphenpenalty=1000
\begin{description*}
	\item[Proficient:]
	Java, Python, JavaScript, React
	\item[Familiar:]
	\CPP, Swift, Objective-C, MIPS Assembly, \LaTeX, Scheme
	\item[Tools:]
	Python SciKit-Learn, NumPy/SciPy, Flask, OpenCV, TensorFlow, Linux, ROS, git, KiCAD
%	\item[Open Source Contributions:]
%	VLC Media Player
%	\item[Languages:] English (Native proficiency); Mandarin Chinese (Bilingual proficiency); French (Professional working proficiency)
\end{description*}
\end{indentsection}

\hrule
\vspace{-0.4em}
\subsection*{Experience}

\begin{itemize}
	\parskip=0.1em

	\item
	\headerrow
		{\textbf{InnoClarity, Full Stack Engineer (Part-time)}}
		{\emph{May 2017-present}}
	\begin{itemize*}
		\item Designed and programmed a portal application, built using React, to allow partners to manage projects using InnoClarity's product
		\item Programmed a backend server to support InnoClarity surveys and the portal application, implemented with Python and Flask
		\item Co-designed a database schema for the product, improving the flexibility and scalability of InnoClarity surveys
	\end{itemize*}

\end{itemize}

\hrule
\vspace{-0.4 em}
\subsection*{Leadership}

\begin{itemize}

	\item
	\headerrow
		{\textbf{Eta Kappa Nu (HKN) Mu Chapter, Bridge Officer}}
		{\emph{May 2017-present}}
	\begin{itemize*}
		\item Tutor students in lower division computer science and electrical engineering classes to help them understand fundamental concepts
		\item Volunteer at outreach events geared towards exposing electrical engineering and computer science to grade schoolers
		\item Take pictures and compile an annual Chapter Report for HKN Nationals
	\end{itemize*}

	\item
	\headerrow
		{\textbf{CalSol Solar Car Team, Telemetry and Data Team}}
		{\emph{Fall 2015-present}}
	\begin{itemize*}
		\item Manage and oversee the telemetry and data team projects
		\item Analyzed race data to better characterize battery pack and model power usage
		\item Designed a battery management system circuit for the onboard secondary battery pack
		\item Optimized CAN telemetry data transmission protocols to reduce data loss and use standardized serialization libraries
		\item Upgraded car telemetry graphical viewer interface, written in Java to improve user experience
	\end{itemize*}

\end{itemize}

%\hrule
%\vspace{-0.4em}
%\subsection*{Projects}
%
%\begin{itemize}
%	\parskip=0.1em
%
%	\item
%	\headerrow
%		{\textbf{Pacman Agent}}
%		{\emph{Spring 2017}}
%	\begin{itemize*}
%		\item Several smaller projects throughout the semester to build an AI agent to play Pacman
%		\item Progressed from using simple search algorithms, to minimax/expectimax agents with alpha/beta pruning, to reinforcement learning, to the use of a simple perceptron to help Pacman navigate the maze while avoiding ghosts
%	\end{itemize*}
%
%	\item
%	\headerrow
%		{\textbf{MIPS Assembler, Linker, and CPU}}
%		{\emph{Fall 2016}}
%	\begin{itemize*}
%		\item Wrote a two-pass assembler that took in MIPS instructions and translated them to machine code
%		\item Implemented a linker in MIPS to take the outputted code from the first part and output the instructions with each relative address resolved
%		\item Finally implement a MIPS CPU in Logism with a 2-stage pipeline
%	\end{itemize*}
%	
%	\item
%	\headerrow
%		{\textbf{Java Swing Text Editor}}
%		{\emph{Spring 2016}}
%	\begin{itemize*}
%		\item Designed and developed text editor from scratch using Java Swing libraries exclusively, writing over 1200 lines of code
%		\item Utilized and designed specific and custom data structures to implement common features such as word wrap, undo/redo, insertion, and cursor
%	\end{itemize*}
%	
%	\item
%	\headerrow
%		{\textbf{Routing Web Application}}
%		{\emph{Spring 2016}}
%	\begin{itemize*}
%		\item Implemented BearMaps, a Google Maps-like web application for the locale of UC Berkeley
%		\item Used data from OpenStreetMap to build database of nodes and paths stored as a graph
%		\item Implemented A* search algorithm to find the shortest path between two nodes
%		\item Utilized Apache Maven to build and run web application on server hosted locally
%	\end{itemize*}
%	
%	\item
%	\headerrow
%		{\textbf{Voice-Control Car}}
%		{\emph{Spring 2016}}
%	\begin{itemize*}
%		\item Constructed circuit using high pass filters and amplifiers to accurately capture voice data
%		\item Implemented PCA to extract features from different voice commands
%		\item Utilized k-means clustering to train microphone data and categorize live voice commands
%	\end{itemize*}
%	
%%	\item
%%        	\headerrow
%%        		{\textbf{Multiplayer Tetris}}
%%        		{\emph{Spring 2015}}
%%        	\begin{itemize*}
%%        		\item Developed a multiplayer Tetris game with a group of two other friends
%%        		\item Utilize Java Swing graphics to paint game environment and Java to implement game back-end
%%        		\item Implement multiplayer functionality through an network chat framework using Java network sockets
%%        	\end{itemize*}
%
%\end{itemize}

\end{document}
